\paragraph{Instructions: } We hope that playing along to the songs will be 
pretty straight-forward even if you don't necessarily know the song already. 
Here are a few instructions that explain some of our conventions.

\paragraph{Chords: } We have selected and edited songs in a way that 
minimizes the need for obscure and hard to play chords. Sometimes, you have 
the option to substitute easier chords. Here are some examples
\begin{itemize}
\item For chords with numbers, you can in a pinch play the chord without the 
number, e.g. A instead of A7.
\item Chords with alternate base notes can be replaced by the normal version 
as well: C instead of C/E.
\item Sustained chords (typically Dsus4, Dsus4, Asus4, Asus2) sound fun and 
are easy to learn, but if you don't want to bother, you'll have to figure out 
whether to substitute the major or minor chord in its place. The fun of 
the sustained chords comes from them being neither major nor minor and keeping 
you on your toes about it.
\item Check out the last page of this songbook for ways to play most of the 
chords occurring in this songbook.
\end{itemize}
